\documentclass[12pt,a4paper]{article}
\usepackage[utf8]{inputenc}
\usepackage[vietnamese]{babel}
\usepackage{geometry}
\usepackage{graphicx}
\usepackage{listings}
\usepackage{xcolor}
\usepackage{hyperref}
\usepackage{fancyhdr}
\usepackage{titlesec}
\usepackage{enumitem}
\usepackage{amsmath}
\usepackage{amssymb}

\geometry{margin=2.5cm}

% Cấu hình màu cho code
\definecolor{codegreen}{rgb}{0,0.6,0}
\definecolor{codegray}{rgb}{0.5,0.5,0.5}
\definecolor{codepurple}{rgb}{0.58,0,0.82}
\definecolor{backcolour}{rgb}{0.95,0.95,0.92}

\lstdefinestyle{mystyle}{
    backgroundcolor=\color{backcolour},
    commentstyle=\color{codegreen},
    keywordstyle=\color{magenta},
    numberstyle=\tiny\color{codegray},
    stringstyle=\color{codepurple},
    basicstyle=\ttfamily\footnotesize,
    breakatwhitespace=false,
    breaklines=true,
    captionpos=b,
    keepspaces=true,
    numbers=left,
    numbersep=5pt,
    showspaces=false,
    showstringspaces=false,
    showtabs=false,
    tabsize=2
}

\lstset{style=mystyle}

% Cấu hình header
\pagestyle{fancy}
\fancyhf{}
\rhead{Cobalt CMS - EC2 Setup Guide}
\lhead{Technest}
\rfoot{Trang \thepage}

% Cấu hình tiêu đề
\titleformat{\section}{\Large\bfseries}{\thesection}{1em}{}
\titleformat{\subsection}{\large\bfseries}{\thesubsection}{1em}{}

\title{\textbf{HƯỚNG DẪN SETUP EC2 CHO COBALT CMS}}
\author{Technest Development Team}
\date{\today}

\begin{document}

\maketitle
\tableofcontents
\newpage

\section{Tổng quan}
Tài liệu này mô tả chi tiết quy trình setup và deploy dự án Cobalt CMS lên Amazon EC2, bao gồm:
\begin{itemize}
    \item Tạo và cấu hình EC2 instance
    \item Cài đặt các dịch vụ cần thiết (Nginx, PM2, Node.js)
    \item Clone source code từ Git repository
    \item Cấu hình database và media files
    \item Build và deploy ứng dụng
    \item Kiểm tra hoạt động
\end{itemize}

\section{Yêu cầu hệ thống}
\begin{itemize}
    \item \textbf{EC2 Instance}: Ubuntu 22.04 LTS (t2.medium trở lên)
    \item \textbf{Storage}: Tối thiểu 20GB
    \item \textbf{Node.js}: Phiên bản 18.x hoặc 22.x
    \item \textbf{Database}: SQLite (built-in) hoặc PostgreSQL/MySQL
    \item \textbf{Web Server}: Nginx
    \item \textbf{Process Manager}: PM2
\end{itemize}

\section{Tạo EC2 Instance}

\subsection{Bước 1: Tạo EC2 Instance}
\begin{enumerate}
    \item Đăng nhập vào AWS Console
    \item Chọn EC2 Service
    \item Click "Launch Instance"
    \item Cấu hình như sau:
    \begin{itemize}
        \item \textbf{Name}: cobalt-cms-production
        \item \textbf{AMI}: Ubuntu Server 22.04 LTS
        \item \textbf{Instance Type}: t2.medium (2 vCPU, 4GB RAM)
        \item \textbf{Key Pair}: Tạo mới hoặc chọn key pair có sẵn
        \item \textbf{Security Group}: Tạo mới với các rule sau:
        \begin{itemize}
            \item SSH (Port 22): 0.0.0.0/0
            \item HTTP (Port 80): 0.0.0.0/0
            \item HTTPS (Port 443): 0.0.0.0/0
            \item Custom TCP (Port 1337): 0.0.0.0/0 (Strapi)
        \end{itemize}
    \end{itemize}
\end{enumerate}

\subsection{Bước 2: Kết nối SSH}
Sau khi instance khởi động, kết nối SSH:
\begin{lstlisting}[language=bash]
ssh -i your-key.pem ubuntu@your-ec2-public-ip
\end{lstlisting}

\section{Setup EC2 Instance}

\subsection{Cập nhật hệ thống}
\begin{lstlisting}[language=bash]
sudo apt update && sudo apt upgrade -y
sudo apt install -y curl wget git unzip
\end{lstlisting}

\subsection{Cài đặt Node.js}
\begin{lstlisting}[language=bash]
# Cài đặt Node.js 18.x
curl -fsSL https://deb.nodesource.com/setup_18.x | sudo -E bash -
sudo apt-get install -y nodejs

# Kiểm tra phiên bản
node --version
npm --version

# Cài đặt PM2 globally
sudo npm install -g pm2
\end{lstlisting}

\subsection{Cài đặt Nginx}
\begin{lstlisting}[language=bash]
# Cài đặt Nginx
sudo apt install -y nginx

# Khởi động và enable Nginx
sudo systemctl start nginx
sudo systemctl enable nginx

# Kiểm tra trạng thái
sudo systemctl status nginx
\end{lstlisting}

\subsection{Cài đặt các dependencies khác}
\begin{lstlisting}[language=bash]
# Cài đặt các package cần thiết
sudo apt install -y build-essential python3
sudo apt install -y sqlite3

# Cài đặt PM2 startup script
pm2 startup
# Chạy lệnh được output từ lệnh trên
\end{lstlisting}

\section{Clone Source Code}

\subsection{Tạo thư mục dự án}
\begin{lstlisting}[language=bash]
# Tạo thư mục dự án
sudo mkdir -p /var/www/cobalt-cms
sudo chown ubuntu:ubuntu /var/www/cobalt-cms
cd /var/www/cobalt-cms
\end{lstlisting}

\subsection{Clone từ Git repository}
\begin{lstlisting}[language=bash]
# Clone source code
git clone https://git.technest.vn/JLAT/cms.git .

# Kiểm tra branch và pull latest
git checkout main
git pull origin main
\end{lstlisting}

\section{Setup Source Code}

\subsection{Cài đặt dependencies}
\begin{lstlisting}[language=bash]
# Cài đặt npm packages
npm install

# Kiểm tra cài đặt
npm list --depth=0
\end{lstlisting}

\subsection{Cấu hình environment}
\begin{lstlisting}[language=bash]
# Copy file environment
cp .env.example .env

# Chỉnh sửa file .env
nano .env
\end{lstlisting}

Cấu hình file \texttt{.env}:
\begin{lstlisting}[language=plaintext]
HOST=0.0.0.0
PORT=1337
APP_KEYS=your-app-keys-here
API_TOKEN_SALT=your-api-token-salt
ADMIN_JWT_SECRET=your-admin-jwt-secret
JWT_SECRET=your-jwt-secret
NODE_ENV=production
DATABASE_CLIENT=sqlite
DATABASE_FILENAME=.tmp/data.db
\end{lstlisting}

\subsection{Cấu hình Server (config/server.ts)}
File \texttt{config/server.ts} cần được cấu hình cho production:
\begin{lstlisting}[language=typescript]
export default ({ env }) => ({
  host: env('HOST', '0.0.0.0'),
  port: env.int('PORT', 1337),
  url: env('PUBLIC_URL', 'https://your-domain.com'),  // ← URL production
  app: {
    keys: env.array('APP_KEYS'),
  },
});
\end{lstlisting}

\subsection{Cấu hình Database (config/database.ts)}
File \texttt{config/database.ts} đã được cấu hình sẵn, nhưng cần đảm bảo environment variables phù hợp:

\subsubsection{Cho SQLite (Production)}
\begin{lstlisting}[language=plaintext]
DATABASE_CLIENT=sqlite
DATABASE_FILENAME=.tmp/data.db
\end{lstlisting}

\subsubsection{Cho PostgreSQL (Production)}
\begin{lstlisting}[language=plaintext]
DATABASE_CLIENT=postgres
DATABASE_HOST=your-db-host
DATABASE_PORT=5432
DATABASE_NAME=your_db_name
DATABASE_USERNAME=your_db_user
DATABASE_PASSWORD=your_db_password
DATABASE_SSL=false
DATABASE_SCHEMA=public
\end{lstlisting}

\subsubsection{Cho MySQL (Production)}
\begin{lstlisting}[language=plaintext]
DATABASE_CLIENT=mysql
DATABASE_HOST=your-db-host
DATABASE_PORT=3306
DATABASE_NAME=your_db_name
DATABASE_USERNAME=your_db_user
DATABASE_PASSWORD=your_db_password
DATABASE_SSL=false
\end{lstlisting}

\subsection{Cấu hình Admin (config/admin.ts)}
File \texttt{config/admin.ts} cần các environment variables sau:
\begin{lstlisting}[language=plaintext]
ADMIN_JWT_SECRET=your-admin-jwt-secret
API_TOKEN_SALT=your-api-token-salt
TRANSFER_TOKEN_SALT=your-transfer-token-salt
ENCRYPTION_KEY=your-encryption-key
FLAG_NPS=false
FLAG_PROMOTE_EE=false
\end{lstlisting}

\subsection{Tạo các keys cần thiết}
\begin{lstlisting}[language=bash]
# Tạo APP_KEYS (cần ít nhất 2 keys)
node -e "console.log(require('crypto').randomBytes(32).toString('base64'))"
node -e "console.log(require('crypto').randomBytes(32).toString('base64'))"

# Tạo JWT secrets
node -e "console.log(require('crypto').randomBytes(32).toString('base64'))"

# Tạo API token salt
node -e "console.log(require('crypto').randomBytes(16).toString('base64'))"

# Tạo encryption key
node -e "console.log(require('crypto').randomBytes(32).toString('base64'))"
\end{lstlisting}

\subsection{File .env hoàn chỉnh cho Production}
\begin{lstlisting}[language=plaintext]
# Server Configuration
HOST=0.0.0.0
PORT=1337
PUBLIC_URL=https://your-domain.com

# App Keys (tạo ít nhất 2 keys)
APP_KEYS=key1,key2

# JWT Secrets
ADMIN_JWT_SECRET=your-admin-jwt-secret-here
JWT_SECRET=your-jwt-secret-here

# API Token
API_TOKEN_SALT=your-api-token-salt-here
TRANSFER_TOKEN_SALT=your-transfer-token-salt-here

# Encryption
ENCRYPTION_KEY=your-encryption-key-here

# Environment
NODE_ENV=production

# Database Configuration (SQLite)
DATABASE_CLIENT=sqlite
DATABASE_FILENAME=.tmp/data.db

# Admin Flags
FLAG_NPS=false
FLAG_PROMOTE_EE=false
\end{lstlisting}

\section{Copy Database và Media Files}

\subsection{Copy Database từ local}
Sử dụng SCP để copy database từ máy local:
\begin{lstlisting}[language=bash]
# Copy database từ local (chạy trên máy local)
scp -i your-key.pem database/data.db ubuntu@your-ec2-public-ip:/var/www/cobalt-cms/.tmp/

# Hoặc copy toàn bộ thư mục database
scp -i your-key.pem -r database/ ubuntu@your-ec2-public-ip:/var/www/cobalt-cms/
\end{lstlisting}

\subsection{Copy Media Files}
\begin{lstlisting}[language=bash]
# Copy thư mục uploads từ local (chạy trên máy local)
scp -i your-key.pem -r public/uploads/ ubuntu@your-ec2-public-ip:/var/www/cobalt-cms/public/

# Hoặc copy toàn bộ thư mục public
scp -i your-key.pem -r public/ ubuntu@your-ec2-public-ip:/var/www/cobalt-cms/
\end{lstlisting}

\subsection{Set permissions}
\begin{lstlisting}[language=bash]
# Set quyền cho database và media files
sudo chown -R ubuntu:ubuntu /var/www/cobalt-cms
sudo chmod -R 755 /var/www/cobalt-cms
sudo chmod 664 /var/www/cobalt-cms/.tmp/data.db
\end{lstlisting}

\section{Build và Deploy}

\subsection{Build ứng dụng}
\begin{lstlisting}[language=bash]
# Build ứng dụng Strapi
npm run build

# Kiểm tra thư mục build
ls -la dist/
\end{lstlisting}

\subsection{Start ứng dụng với PM2}
\begin{lstlisting}[language=bash]
# Start ứng dụng
pm2 start npm --name "cobalt-cms" -- start

# Kiểm tra trạng thái
pm2 status
pm2 logs cobalt-cms

# Save PM2 configuration
pm2 save
\end{lstlisting}

\section{Cấu hình Nginx}

\subsection{Tạo Nginx configuration}
\begin{lstlisting}[language=bash]
sudo nano /etc/nginx/sites-available/cobalt-cms
\end{lstlisting}

Nội dung file configuration:
\begin{lstlisting}[language=nginx]
server {
    listen 80;
    server_name your-domain.com www.your-domain.com;

    # Redirect HTTP to HTTPS (nếu có SSL)
    # return 301 https://$server_name$request_uri;

    location / {
        proxy_pass http://localhost:1337;
        proxy_http_version 1.1;
        proxy_set_header Upgrade $http_upgrade;
        proxy_set_header Connection 'upgrade';
        proxy_set_header Host $host;
        proxy_set_header X-Real-IP $remote_addr;
        proxy_set_header X-Forwarded-For $proxy_add_x_forwarded_for;
        proxy_set_header X-Forwarded-Proto $scheme;
        proxy_cache_bypass $http_upgrade;
    }

    # Serve static files
    location /uploads/ {
        alias /var/www/cobalt-cms/public/uploads/;
        expires 1y;
        add_header Cache-Control "public, immutable";
    }

    # Admin panel
    location /admin {
        proxy_pass http://localhost:1337;
        proxy_http_version 1.1;
        proxy_set_header Upgrade $http_upgrade;
        proxy_set_header Connection 'upgrade';
        proxy_set_header Host $host;
        proxy_set_header X-Real-IP $remote_addr;
        proxy_set_header X-Forwarded-For $proxy_add_x_forwarded_for;
        proxy_set_header X-Forwarded-Proto $scheme;
        proxy_cache_bypass $http_upgrade;
    }
}
\end{lstlisting}

\subsection{Enable site và restart Nginx}
\begin{lstlisting}[language=bash]
# Enable site
sudo ln -s /etc/nginx/sites-available/cobalt-cms /etc/nginx/sites-enabled/

# Test configuration
sudo nginx -t

# Restart Nginx
sudo systemctl restart nginx
\end{lstlisting}

\section{Test và Kiểm tra}

\subsection{Kiểm tra ứng dụng}
\begin{lstlisting}[language=bash]
# Kiểm tra PM2 processes
pm2 status
pm2 logs cobalt-cms

# Kiểm tra port 1337
sudo netstat -tlnp | grep :1337

# Kiểm tra Nginx
sudo systemctl status nginx
\end{lstlisting}

\subsection{Test từ browser}
\begin{itemize}
    \item \textbf{Main site}: \texttt{http://your-ec2-public-ip}
    \item \textbf{Admin panel}: \texttt{http://your-ec2-public-ip/admin}
    \item \textbf{API}: \texttt{http://your-ec2-public-ip/api}
\end{itemize}

\subsection{Test API endpoints}
\begin{lstlisting}[language=bash]
# Test API health
curl http://localhost:1337/api/health

# Test từ external
curl http://your-ec2-public-ip/api/health
\end{lstlisting}

\section{Monitoring và Maintenance}

\subsection{PM2 Monitoring}
\begin{lstlisting}[language=bash]
# Xem logs real-time
pm2 logs cobalt-cms --lines 100

# Restart ứng dụng
pm2 restart cobalt-cms

# Reload ứng dụng
pm2 reload cobalt-cms

# Stop ứng dụng
pm2 stop cobalt-cms
\end{lstlisting}

\subsection{System Monitoring}
\begin{lstlisting}[language=bash]
# Kiểm tra disk usage
df -h

# Kiểm tra memory usage
free -h

# Kiểm tra CPU usage
htop

# Kiểm tra Nginx access logs
sudo tail -f /var/log/nginx/access.log
\end{lstlisting}

\section{Troubleshooting}

\subsection{Các vấn đề thường gặp}

\subsubsection{Port 1337 không accessible}
\begin{lstlisting}[language=bash]
# Kiểm tra firewall
sudo ufw status

# Mở port nếu cần
sudo ufw allow 1337
\end{lstlisting}

\subsubsection{Database connection error}
\begin{lstlisting}[language=bash]
# Kiểm tra file database
ls -la .tmp/data.db

# Kiểm tra permissions
sudo chown ubuntu:ubuntu .tmp/data.db
sudo chmod 664 .tmp/data.db
\end{lstlisting}

\subsubsection{Nginx 502 Bad Gateway}
\begin{lstlisting}[language=bash]
# Kiểm tra ứng dụng có chạy không
pm2 status

# Kiểm tra logs
pm2 logs cobalt-cms

# Restart ứng dụng
pm2 restart cobalt-cms
\end{lstlisting}

\section{Backup và Recovery}

\subsection{Backup Database}
\begin{lstlisting}[language=bash]
# Tạo backup database
cp .tmp/data.db .tmp/data.db.backup.$(date +%Y%m%d_%H%M%S)

# Backup toàn bộ thư mục
tar -czf cobalt-cms-backup-$(date +%Y%m%d_%H%M%S).tar.gz \
    .tmp/ public/uploads/ config/ src/
\end{lstlisting}

\subsection{Recovery từ backup}
\begin{lstlisting}[language=bash]
# Restore database
cp .tmp/data.db.backup.20241201_143000 .tmp/data.db

# Restore từ archive
tar -xzf cobalt-cms-backup-20241201_143000.tar.gz
\end{lstlisting}

\section{Kết luận}

Quy trình setup EC2 cho Cobalt CMS đã hoàn thành với các bước chính:
\begin{enumerate}
    \item Tạo và cấu hình EC2 instance
    \item Cài đặt các dịch vụ cần thiết
    \item Clone và setup source code
    \item Copy database và media files
    \item Build và deploy ứng dụng
    \item Cấu hình Nginx reverse proxy
    \item Test và monitoring
\end{enumerate}

\subsection{Lưu ý quan trọng}
\begin{itemize}
    \item Luôn backup database trước khi deploy
    \item Kiểm tra permissions cho các file và thư mục
    \item Monitor logs để phát hiện vấn đề sớm
    \item Cập nhật security groups và firewall rules
    \item Sử dụng PM2 để quản lý process hiệu quả
\end{itemize}

\subsection{Liên hệ hỗ trợ}
Nếu gặp vấn đề trong quá trình setup, vui lòng liên hệ:
\begin{itemize}
    \item \textbf{Email}: dev@technest.vn
    \item \textbf{Slack}: \#cobalt-cms-support
    \item \textbf{Documentation}: \url{https://docs.technest.vn/cobalt-cms}
\end{itemize}

\end{document}